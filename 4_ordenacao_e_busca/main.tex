\section{Ordenação e Busca}

\subsection{Insertion Sort}
Complexidade O($n^2$);
\begin{verbatim}
void insertionSort (vector<int> &array, int size){
    int i, key, j;
    for (i=1; i<size; i++){
        key = array[i];
        j = i-1;
        while (j>=0 && array[j] > key){
            array[j+1] = array[j];
            j=j-1;
        }
        array[j+1] = key;
    }
}
\end{verbatim}

\subsection{Merge Sort}
Complexidade O(n log(n))
\begin{verbatim}
void merge(vector<int> &array, int left, int mid, int right){
    vector<int> firstHalf (mid-left+1);
    vector<int> secondHalf (right-mid);
 
    for (int i=0; i < mid-left+1; i++)  
        firstHalf[i] = array[left+i];
    for (int i=0; i < right-mid; i++)
        secondHalf[i] = array[mid + i + 1];

    int j=0, k=0, i=left; 

    while (j<mid-left+1 && k < right - mid){
        if (firstHalf[j] <= secondHalf[k])
            array[i++] = firstHalf[j++];
        else
            array[i++] = secondHalf[k++];
    }

    while (j < mid-left+1)
        array[i++] = firstHalf[j++];
    while (k < right-mid)
        array[i++] = secondHalf[k++];
}
 

void mergeSort(vector<int> &array, int begin, int end){
    if (begin >= end)
        return;
 
    int mid = (begin + end)/2;
    mergeSort(array, begin, mid);
    mergeSort(array, mid + 1, end);
    merge(array, begin, mid, end);
}
\end{verbatim}

\subsection{STL Sort}
    Ordena os elementos de um vector em complexidade O(n*log(n))
\begin{verbatim}
    sort (Iterator_begin, Iterator_first);
    //Geralmente utilizado da seguinte forma,
    sort (vector_name.begin(), vector_name.end());

    - Para se fazer a ordenação em ordem decrescente usa-se:
    sort (vector_name.begin(), vector_name.end(), greater<int>());

    - Caso queiramos fazer nossa própria função de ordenação, podemos utilizar:
    sort(vector_name.begin(), vector_name.end(), function_name);

    bool (vector_type i1, vector_type i2){
        //CÓDIGO AQUI
        return true; //Se a ordem já estiver certa.
        return false; //Se a ordem estiver errada.
    }
\end{verbatim}

\subsection{Binary Search}
Para a realização da busca binária, é necessário que o vetor esteja inicialmente ordenado. Existem comandos da STL que facilitam a realização do processo.\\

-- UTILIZANDO VECTOR --
1. lower\_bound: retorna um iterator da primeira posição do vetor com um valor não menor que o desejado.
\begin{verbatim}
auto it = lower_bound(vec.begin(), vec.end(), wanted);
\end{verbatim}

2. upper\_bound: retorna um iterator apontando para o primeiro elemento maior que o valor desejado.
\begin{verbatim}
auto it = upper_bound(vec.begin(), vec.end(), wanted);
\end{verbatim}
- Nos casos em que todos os valores sejam menores que "wanted" a função retorna um iterador apontando para o último elemento do vector + 1.\\

-- UTILIZANDO SET -- (utilizar as funções abaixo, caso contrário existem casos O(n))

1. lower\_bound: retorna um iterator da primeira posição do vetor com um valor não menor que o desejado.
\begin{verbatim}
auto it = set_name.lower_bound(wanted);
\end{verbatim}

2. upper\_bound: retorna um iterator apontando para o primeiro elemento maior que o valor desejado.
\begin{verbatim}
auto it = set_name.upper_bound(wanted);
\end{verbatim}
- Nos casos em que todos os valores sejam menores que "wanted" a função retorna um iterador apontando para o último elemento do vector + 1.

Nos casos em que se faz necessário codar a busca binária, segue a função:
\begin{verbatim}
    int binarySearch(vector<int>& arr, int low, int high, int wanted) 
    { 
        while (low <= high) { 
            int mid = (high + low)/2; 

            if (arr[mid] == wanted) {return mid;} 
            if (arr[mid] < wanted) {low = mid + 1;}  
            else {high = mid - 1;} 
        } 
        return -1; 
    } 
\end{verbatim}

