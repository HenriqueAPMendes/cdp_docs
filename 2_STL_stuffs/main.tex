\section{STL Stuffs}
As estruturas Vector, List e Deque armazenam os elementos em ordem linear,
seguindo a estrutura na qual foram inseridos.

\subsection{Pair}
Combinação de dois valores que podem (ou não) possuir tipos diferentes.
\begin{verbatim}
    pair <type1, type2> Pair_name 
    //Criação de um pair.

    pair <type1, type2> Pair_name (value1, value2) 
    //Criação do um pair com valores já escritos.

    pair pair_name (1, 'a') 
    //Criação de um pair sem pré-determinar os tipos.

    g2 = make_pair(1, 'a')
    g2 = {1, 'a'}
    //Atribuição de um pair (ideal para uso em vectors).
\end{verbatim}

\subsection{Vector}
Vetor de alocação dinâmica, que possui os seguintes comandos principais:
\begin{verbatim}
    begin()
    //Returns an iterator pointing to the first element.

    end() 
    //Returns an iterator pointing to the last element.

    size() 
    //Returns the number of elements in the vector.

    empty() 
    //Returns whether the container is empty.

    front() 
    //Returns a reference to the first element in the vector.

    nome_vec[i] 
    //Returns a reference to the element at position ‘i’ in the vector.

    push_back(element) 
    //Push the elements into a vector from the back. (copia o elemento)

    pop_back() 
    //Remove elements from a vector from the back.

    insert(index, val) 
    //Inserts new elements before the element at the specified position
    //No casos envolvendo inserção de um caracter em uma string, deve-se usar 
    str.insert(index, 1, 'char');

    clear() 
    //Destroy all the elements of the vector.

    erase(posIn, posFinal) 
    //Remove elements from a container from the specified range (used with iterators).
    - Feito em tempo O(n). Set faz esse processo em tempo O(1).

    emplace(position /*Iterator*/, element); 
    //Insert a element at the position (constrói o elemento).

    emplace_back(value); 
    //Insert a element at the end of the vector.

    auto it = find(vetor.begin(), vetor.end(), valorProcurado);
    //Encontra o iterator da posição que tem valor = valorProcurado.
    //Retorna vetor.end() caso não encontre.
\end{verbatim}

\subsection{List}


\subsection{Deque}
    Lista que possui dois fins (um no começo e outro no final).
    - Parecido com a fila, mas faz a inserção/remoção do começo/fim em O(1).
\begin{verbatim}
    deque_name.insert (iterator, value);
    //Insere o elemento na posição do itarator.

    deque_name.push_front(value);
    deque_name.push_back(value);
    //Adiciona um elemento no começo/fim do deque.

    deque_name.pop_front();
    deque_name.pop_back();
    //Retira um elemento do começo/fim do deque. (possui tipo void).

    deque_name.clear();
    //Exclui todos os elementos do deque.

    deque_name.erase(iterator1, iterator2);
    //Exclui os elementos do intervalo delimitado.

    deque_name.empty();
    //Verifica se o deque está vazio.

    deque_name.front();
    deque_name.back();
    //Acessa os elementos do começo/fim do deque.

    deque_name.size();
    //Verifica o tamanho do deque.
\end{verbatim}

\subsection{Stack}
    Feito de acordo com os comandos padrões do vector.

\subsection{Queue}

\subsection{Priority Queue}

\subsection{Map}
\begin{verbatim}
    for(auto& it:map_name)
    //Faz a iteração no map.
\end{verbatim}

\subsection{Set}    
    erase(posIn, posFinal) 
    //Remove elements from a container from the specified range (used with iterators).
    - Feito em tempo O(1).

\subsection{MultiSet} 

\pagebreak