\section{Template e Compilação}
\subsection{Template}
\begin{enumerate}
    \item Diretório C++ que contém quase todos os comandos necessários
        \begin{verbatim}
        #include <bits/stdc++.h>
        using namespace std;
        \end{verbatim}
    \item Comando para deixar operações do C++ como cin/cout mais rápidas.
        \begin{verbatim}
        cin.tie(0);
        ios_base::sync_with_stdio(0);
        \end{verbatim}
    \item Template:
    \begin{verbatim}
        #include <bits/stdc++.h>
        using namespace std;

        int main(){
            cin.tie(0);
            ios_base::sync_with_stdio(0);

            return 0;
        }
    \end{verbatim}
    
\end{enumerate}

\subsection{Compilação}
    \begin{enumerate}
        \item Entrar no diretório que está o arquivo .cpp
        \item Compilar (substituir "A.cpp" pelo arquivo .cpp em questão): 
        \begin{verbatim}
            g++ -g -static -Wall -O2 A.cpp
        \end{verbatim}
        \item Se Linux: executar o comando "./a.out $<$ tmp.in", onde tmp.in é o arquivo com as entradas\\
    \end{enumerate}
\pagebreak
%%%%%%%%%%%%%%%%%%%%%%%%%%%%%%%%%%%%%%%%%%%%%%%%%%%%%%%%%%%%%%%%%%%%%%%%%%%%%%%%%%%%%%%%%

