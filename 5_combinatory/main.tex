\section{Combinatória}

\subsection{Princípio da Contagem}
    \subsubsection{Princípio da Multiplicação}
        Se um evento A pode ocorrer de M maneiras diferentes e um evento B pode ocorrer de N maneiras diferentes,
        então existem $M*N$ combinações desses elementos (calças e camisetas).

    \subsubsection{Lei da Multiplicação}
        Se um evento A possui probabilidade M e um evento B possui probabilidade N,
        então a probabilidade de A e B ocorrerem é de $A*B$.

    \subsubsection{Princípio da Adição}
        Se um evento A possui probabilidade M e um evento B possui probabilidade N,
        então a probabilidade de A ou B ocorrerem é de $A+B$.

\subsection{Permutação}
    Dados um conjunto com n caixas iguais, queremos saber as diferentes formas de organizar n bolas distintas $(qtdBolas = qtdCaixas)$
    
    É descobrir todas as permutações por meio de uma função da STL de complexidade $O(n*n!)$
\begin{verbatim}
bool next_permutation (Iterator begin, Iterator end);
//Rearranja os elementos na próxima permutação, sendo essa a menor 
possível maior que a atual (no quesito lexicográfico); retornando 
se a operação foi bem sucedida ou não.    
\end{verbatim}
    
    \subsubsection{Permutação sem repetição}
    \begin{equation}
        \centering
        P(n) = n!
    \end{equation}

    \subsubsection{Permutação com repetição}
    \begin{equation}
        \centering
        P(n, k, l, ...) = n!/k!l!...
    \end{equation}
    k, l,... representam a quantidade de vezes que a bola (k, l, ...) se repete

\subsection{Combinação}
    Permutação onde a quantidade de bolas (n) é maior que a quantidade de caixas (k), além de que a 
    ordem das bolas não faz diferença.\newline
    \begin{equation}
        \centering
        C(n, k) = \frac{n!}{k!(n-k)!}
    \end{equation}

\subsection{Arranjo Simples}
    Permutação onde a quantidade de bolas (n) é maior que a quantidade de caixas (p), além de que a 
    ordem das bolas faz diferença.\newline
\begin{equation}
    \centering
    C(n, k) = \frac{n!}{(n-p)!}
\end{equation}
        

