\section{Programação Dinâmica}
    
    \subsection{Fibonacci}
        Exemplo clássico de programação dinâmica, reduzindo a complexidade do Fibonacci para O(n)
\begin{verbatim}
int Fibonacci(vector<int> &MEMO, int n){
    if (MEMO[n] != -1)
        return MEMO[n];
    if (n == 1 || n==2){
        MEMO [n] = 1;
        return MEMO[n];
    }
    MEMO[n] = Fibonacci(MEMO, n-1) + Fibonacci(MEMO, n-2);
    return MEMO[n];
}

vector<int> MEMO(n+1, -1);
int result = Fibonacci(MEMO, n);    
\end{verbatim}

\subsection{Problema da mochila}
    Selecionar um conjunto de itens (itens que possuem pesos e valores) de modo que o somatório dos valores seja máximo, respeitando a capacidade da mochila.
\begin{verbatim}
int mochila01 (int W, int N, vector<int>& peso, vector<int>& valor){
    int MEMO [N+1][W+1];
    for (int i=0; i<=N; i++)
        for (int j=0; j<=W; j++)
            MEMO[i][j] = 0;

    for (int i=1; i<=N; i++){
        for (int j=1; j<=W; j++){
            if (peso[i] > j)
                MEMO[i][j]=MEMO[i-1][j];
            else
                MEMO[i][j] = max(MEMO[i-1][j], MEMO[i-1][j-peso[i]]+valor[i]);
        }
    }
    return (MEMO[N][W]);
}
\end{verbatim}

\subsection{Longest Common Subsequence}
Dada duas strings X e Y, calcula a maior sequência comum entre ambas.

\begin{verbatim}
//Essa função pode ser alterada para calcular a sequência de qualquer tipo.
//Para se descobrir essa subsequência, deve-se criar uma matriz auxiliar de setas.
int lcs (string X, string Y, int m, int n){
    vector<vector<int>> MEMO (m+1, vector<int>(n+1, 0));

    for (int i=0; i<=m; i++){
        for (int j=0; j<=n; j++){
            if (i == 0 || j==0)
                MEMO[i][j] = 0;
            else if (X[i-1] == Y[j-1])
                MEMO[i][j] = MEMO[i-1][j-1] + 1;
            else
                MEMO[i][j] = max(MEMO[i-1][j], MEMO[i][j-1]);
        }
    }

    return (MEMO[m][n]);
}
\end{verbatim}
\pagebreak
%%%%%%%%%%%%%%%%%%%%%%%%%%%%%%%%%%%%%%%%%%%%%%%%%%%%%%%%%%%%%%%%%%%%%%%%%%%%%%%%%%%%%%%%%

