\section{Geometria}

\subsection{Convex Hull}

Calcula o fecho convexo de um conjunto de pontos em $\mathcal{O} (n\log n)$.
Para incluir ou não pontos colineares, basta mudar INCLUDE\_COLLINEAR

\begin{verbatim}
    #define CLOCKWISE -1
    #define COUNTERCLOCKWISE 1
    #define INCLUDE_COLLINEAR 1
    
    struct pt {
        int x, y;
        bool operator == (pt const& t) const {
            return x == t.x && y == t.y;
        }
    };
    
    struct vec {
        int x, y, z;
    };
    
    #define vp vector<pt>
    
    vec cross(vec v1, vec v2){
        int x = v1.y*v2.z - v1.z*v2.y;
        int y = v1.z*v2.x - v1.x*v2.z;
        int z = v1.x*v2.y - v1.y*v2.x;
        return {x,y,z};
    }
    
    int dist2(pt p1, pt p2) {
        int dx = p1.x - p2.x;
        int dy = p1.y - p2.y;
        return dx*dx + dy*dy;
    }
    
    int orientation(pt a, pt b, pt c) {
        vec v1 = {b.x-a.x, b.y-a.y, 0};
        vec v2 = {c.x-b.x, c.y-b.y, 0};
        vec v = cross(v1, v2);
        if (v.z < 0) return CLOCKWISE;
        if (v.z > 0) return COUNTERCLOCKWISE;
        return 0;
    }
    
    bool clock_wise(pt a, pt b, pt c) {
        int o = orientation(a, b, c);
        return o < 0 || (INCLUDE_COLLINEAR && o == 0);
    }
    bool collinear(pt a, pt b, pt c) { return orientation(a, b, c) == 0; }
    
    vp convex_hull(vector<pt>& a) {
        int n = a.size();
        pt first = *min_element(a.begin(), a.end(), [](pt a, pt b) {
            return ii(a.y, a.x) < ii(b.y, b.x);
        });
    
        sort(a.begin(), a.end(), [first](pt a, pt b) {
            int o = orientation(first, a, b);
            if (o == 0)
                return dist2(first, a) < dist2(first, b);
            return o == CLOCKWISE;
        });
        
        if (INCLUDE_COLLINEAR) {
            int i = n-1;
            while (i >= 0 && collinear(first, a[i], a.back())) i--;
            reverse(a.begin()+i+1, a.end());
        }
    
        vector<pt> ans;
        for (auto p : a) {
            while (ans.size() > 1 && !clock_wise(ans[ans.size()-2], ans.back(), p))
                ans.pop_back();
            ans.push_back(p);
        }
    
        if (!INCLUDE_COLLINEAR && ans.size() == 2 && ans[0] == ans[1])
            ans.pop_back();
    
        return ans;
    }
    
    
\end{verbatim}

