\section{Grafos}

\subsection{Lista de adjacências} 

\par A estrutura utilizada é tal que as adjacências são da forma 
g[u] = $\left\{w,v\right\}$. 
O peso está em primeiro para facilitar um eventual sort.

\begin{verbatim}
        vector<vector<ii>> g(MAX);

        // para adicionar aresta que conecta $u$ e $v$ com custo $w$ fazer:
        g[u].push_back({w,v}); 
        g[v].push_back({w,u}); 
    \end{verbatim}

\subsection{DFS:}
\par Chamar uma dfs para cada vértice, da forma:
\begin{verbatim}
    for (int u = 0; u < n; u++) 
        if (!vis[u]) dfs(u, u+1);
\end{verbatim}
Complexidade : $\mathcal{O}(V+E)$
    \begin{verbatim}
void dfs(int u, int val){
    vis[u] = val;
    for (auto &v : g[u])
        if (!vis[v.second]) dfs(v.second,val);
    
}
    \end{verbatim}

\subsection{BFS}
\par Salva as distâncias a partir do vértice de início s.\\
Complexidade : $\mathcal{O}(V+E)$
\begin{verbatim}
    void bfs(int s){
        d[s] = 0;
        queue<int> q; q.push(s);
        while(!q.empty()){
            int u = q.front(); q.pop();
            for (auto &v : g[u])
                if (d[v.second] == INF){
                    d[v.second] = d[u]+1;
                    q.push(v.second);
                }
        }
    }
    
\end{verbatim}

\subsection{Ordenação Topológica} 
Atribui rótulos aos vértices de um grafo DAG de modo que para toda aresta (u, v), 
u.rótulo $<$ v.rótulo. Utiliza uma DFS com pequena variação:
\par ATENÇÃO: esse código salva a ordem topológia reversa.
\begin{verbatim}
    vector<int> ts; // topological sort
    
    void dfs(int u, int val){
        if (vis[u]) return;
        vis[u] = val;
        for (auto &v : g[u])
            if (!vis[v.second]) dfs(v.second,val);
        
        ts.push_back(u);   
    }
\end{verbatim}

\subsection{Componentes Conexas}
\par Código que conta a quantidade de componentes conexas e marca cada vértice com o valor de sua componente.
\begin{verbatim}
    int ans = 0;
    for (int u = 0; u < n; u++)
        if (!vis[i]) dfs(u, ++ans);
        
    
\end{verbatim}

\subsection{Checagem de grafo bipartido}
\par Para checar se um grafo é bipartido, "colorimos" ele com uma BFS modificada.
\begin{verbatim}
    bool bfs(){
        color[s] = 0;
        bool isBipartite = true;
        queue<int> q; q.push(s);
        while(!q.empty() && isBipartite){
            int u = q.front(); q.pop();
            for (auto &v : g[u]){
                if (color[v.second] == INF){
                    color[v.second] = 1-color[u];
                    q.push(v.second);
                }
                else if (color[v.second] == color[u]{
                    isBipartite = false; break;
                }
            }
        }
        return isBipartite;
    }
\end{verbatim}

\subsection{Árvores Geradoras Mínimas - MST}
\subsubsection{Kruskal}
Utiliza MUF para unir as arestas e seus vértices.
Complexidade: $\mathcal{O}(E\log V)$
\begin{verbatim}
    int kruskal(){
        int cost = 0, u, v, w;
        for (int i = 0; i < n; i++) {_p[i]=i; _rank[i]=0;}
        sort(edges.begin(), edges.end());
        for (auto &e : edges){
            u = e.second.first;
            v = e.second.second;
            w = e.first;
            if (_find(u) != _find(v)){
                cost += w;
                _union(u, v);
            }
        }
        return cost;
    }

\end{verbatim}
\subsubsection{Prim}
\par Utiliza fila de prioridade e lista de adjacência para escolher novo vértice com aresta de menor peso.
Complexidade: $\mathcal{O}(E\log V)$

\begin{verbatim}
    vi taken(MAX);

    int prim(){
        priority_queue<ii> pq;
        taken[0] = 1;
        for (auto &v : g[0])
            if (!taken[v.second]) pq.push({-v.first, -v.second});  
    
        int cost = 0;
        while(!pq.empty()){
            ii front = pq.top(); pq.pop();
            int w = -front.first;
            int u = -front.second;
            if (!taken[u]) {
                cost += w;
                taken[u] = 1;
                for (auto &v : g[u])
                    if (!taken[v.second]) pq.push({-v.first, -v.second});  
            }
        }
        return cost;
    }
    
\end{verbatim}
\subsection{Caminhos mínimos}\

\subsubsection{Dijkstra}
Geralmente INF é utilizado como 2123456789012345
Complexidade: $\mathcal{O}(E\log V)$

\begin{verbatim}
    void dijkstra(int s){
        int d, u, v;
        for (int i = 0; i < n; i++) dist[i] = INF; dist[s] = 0;
        priority_queue<ii, vector<ii>, greater<ii>> pq;
        pq.push({0,s});
        while(!pq.empty()){
            d = pq.top().first;
            u = pq.top().second;
            pq.pop();
            if (d > dist[u]) continue;

            for (auto &x : adj[u]){
                d = x.first;
                v = x.second;
                if (dist[v] > dist[u] + d){
                    dist[v] = dist[u] + d; 
                    pq.push({dist[v], v});
                }   
            }
        }
    }

\end{verbatim}

\subsubsection{Belmann-Ford}
\par Pode ser utilizado para detectar ciclos negativos.
Complexidade: $\mathcal{O}(EV)$

\begin{verbatim}
    void belmann_ford(int s){
        for (int i = 0; i < V; i++) dist[i] = INF;
        dist[s] = 0;
        for (int i = 0; i < V; i++)
            for (int u = 0; u < V; u++)
                for (auto &v : adj[u])
                    dist[v.second] = min(dist[v.second], dist[u]+v.first);
    }
\end{verbatim}

\par Para detectar ciclos negativos, executar mais uma varredura, de forma que o código 
fique da seguinte maneira:

\begin{verbatim}
    void belmann_ford(int s){
        for (int i = 0; i < V; i++) dist[i] = INF;
        dist[s] = 0;
        for (int i = 0; i < V; i++)
            for (int u = 0; u < V; u++)
                for (auto &v : adj[u])
                    dist[v.second] = min(dist[v.second], dist[u]+v.first);
    }

    int main(){
        ...

        bellmann_ford(s);

        bool hasNegativeCycle = false;
        for (int u = 0; u < V; u++)
            for (auto &v : adj[u])
                if (dist[v].second > dist[u]+v.first)
                    hasNegativeCycle = true;

        if (hasNegativeCycle) cout << "CYCLE" << endl;
        else cout << "NO CYCLE" << endl;
        ...
    }
\end{verbatim}


\subsubsection{Floyd Warshall} calcula as menores distâncias entre todos os vértices de um grafo.
    \begin{verbatim}
void floyd_warshall(int graph[][N], int N) {
    int dist[N][N], i, j, k;

    for (i = 0; i < N; i++)
        for (j = 0; j < N; j++)
            dist[i][j] = graph[i][j];

    for (k = 0; k < N; k++) 
        for (i = 0; i < N; i++) 
            for (j = 0; j < N; j++) 
                if (dist[i][k] + dist[k][j] < dist[i][j])
                    dist[i][j] = dist[i][k] + dist[k][j];
}
    \end{verbatim}

\subsubsection{k-th ancestral} Possui o objetivo de encontrar o k-ésimo ancestral de um nó V em complexidade O(log k)
    \begin{verbatim}
class BinaryLifiting{
    public: 
        int num_nodes;
        vector<vector<int>> ancestor;
        vector<bool> visited;
        vector<vector<pair<int, int>>> g;

        BinaryLifiting(int V): 
            num_nodes(V), 
            ancestor(V + 1, vector<int>(LOG, -1)),
            g(MAX), 
            visited(MAX, false){}

        void dfs(int node, int parent){
            ancestor[node][0] = parent;
            visited[node] = true;
            for (pair<int, int> neighbor : g[node]) {
                if (!visited[neighbor.second]) {
                    dfs(neighbor.second, node);
                }
            }
        }

        void preprocess(){
            dfs(1, -1);
            for (int j = 1; j < LOG; j++) {
                for (int i = 1; i <= num_nodes; i++) {
                    if (ancestor[i][j-1] != -1)
                        ancestor[i][j] = ancestor[ancestor[i][j-1]][j-1];
                }
            }
        }

        int findKthAncestor(int node, int K){
            for (int i = LOG - 1; i >= 0; i--) {
                if (K & (1 << i)) {
                    if (ancestor[node][i] == -1)
                        return -1;
                    node = ancestor[node][i];
                }
            }
            return node;
        }

        void addEdge(int u, int v, int w) {
            g[u].push_back({w, v});
            g[v].push_back({w, u});
        }
}
    \end{verbatim}
\pagebreak
%%%%%%%%%%%%%%%%%%%%%%%%%%%%%%%%%%%%%%%%%%%%%%%%%%%%%%%%%%%%%%%%%%%%%%%%%%%%%%%%%%%%%%%%%
