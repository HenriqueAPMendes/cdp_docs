\section{Vetores}

\subsection{Inicializacao}
\subsubsection{1D}
\begin{itemize}
    \item \lstinline{vector<int> v(size, val);}
\end{itemize}

\subsubsection{2D}
\begin{itemize}
    \item \lstinline{vector<vector<int> > a(rows, vector<int>(columns, 0));}
\end{itemize}

\subsection{Difference Array}
\par Utilizado para somar ou subtrair valores de um vetor dado um range e um comportamento específico.
\begin{itemize}
    \item Estrutura inicial: array a[]
    \item Operações de range O(1) ($\pm d$)
    \item Restaurar os valores do array: O(n)
\end{itemize}

\begin{verbatim}
    delta = 0;
    for (int i = 0; i < MAX; i++){
        delta += diff[i]; // array de diferenças
        a[i] += delta;
    }
\end{verbatim}

\subsubsection{Difference Array 2D}
\begin{itemize}
    \item $\lstinline{dcol}$ é um vetor de acumulado para as colunas, responsável por garantir o comportamento do range de somas.
\end{itemize}

\begin{verbatim}
    for (int i = 0; i < MAX; i++){
        int delta = 0;
        for (int j = 0; j < MAX; j++){
            dcol[j] += diff[i][j];
            delta += dcol[j];
            a[i][j] += delta;
        }
    }
\end{verbatim}

\subsection{Prefix Sum}
    Verifica se é possível determinado valor de soma ocorrer considerando qualquer subintervalor de um vetor.
    - Possui complexidade O(n*log n).

\begin{verbatim}
bool prefix_sum(vector<int>& numbers, int size){
    int wanted = 0; //Can be any sum wanted.
    set<long long> sumSet; long long sum = 0; 

    for (int i=0; i<size; i++){
        sum += numbers[i];
        if (sum == wanted || sumSet.find(sum - wanted) != sumSet.end())
            return true;

        sumSet.insert(sum);
    }
    
    return false;
}
\end{verbatim}






\subsection{Segment tree}
\par Extensivamente utilizada para fazer operações em intervalos de vetores, tanto atualização como consultas sao feitas em $O(\log n)$. \\
Um teste com $n$ operações de consulta implementadas linearmente criariam um algoritmo $O(n^2)$, enquanto aplicando SegTree, $n$ operações geram um algoritmo $O(n\log n)$.
\par OBS: a árvore é indexada com o pai em 1, ou seja, chamar \lstinline{_build(1, 0, n-1)}, \lstinline{_query(1, 0, n-1, i, j)}, \lstinline{_update(1, 0, n-1, i, val)}

\subsubsection{SegTree Aditiva}
\begin{verbatim}

    int a[MAX], tree[4*MAX];

    void build(int node, int start, int end){
        // Leaf node
        if(start == end) tree[node] = a[start];        
        else{
            int mid = (start + end) >> 1;
            
            build(2*node, start, mid);
            build(2*node+1, mid+1, end);
            
            tree[node] = tree[2*node] + tree[2*node+1];
        }
    }
    
    void update(int node, int start, int end, int idx, int val){
        // Leaf node
        if(start == end){
            A[idx] += val;
            tree[node] += val;
            return;
        }
    
        int mid = (start + end) >> 1;
        
        if(start <= idx and idx <= mid) 
            update(2*node, start, mid, idx, val); // left child
        else 
            update(2*node+1, mid+1, end, idx, val); // right child
        
        tree[node] = tree[2*node] + tree[2*node+1];
        
    }

    int query(int node, int start, int end, int l, int r){
        // range completely outside
        if(r < start or end < l) return 0; // sum neutral

        // range completely inside
        if(l <= start and end <= r) return tree[node];
        
        // range partially inside and partially outside
        int mid = (start + end) >> 1;
        int p1 = query(2*node, start, mid, l, r);
        int p2 = query(2*node+1, mid+1, end, l, r);
        return (p1 + p2);
    }
\end{verbatim}

\subsubsection{SegTree Multiplicativa}
\begin{verbatim}

    int a[MAX], tree[4*MAX];

    void build(int node, int start, int end){
        // Leaf node
        if(start == end) tree[node] = a[start];        
        else{
            int mid = (start + end) >> 1;
            
            build(2*node, start, mid);
            build(2*node+1, mid+1, end);
            
            tree[node] = tree[2*node] * tree[2*node+1];
        }
    }
    
    void update(int node, int start, int end, int idx, int val){
        // Leaf node
        if(start == end){
            A[idx] += val;
            tree[node] += val;
            return;
        }
    
        int mid = (start + end) >> 1;
        
        if(start <= idx and idx <= mid) 
            update(2*node, start, mid, idx, val); // left child
        else 
            update(2*node+1, mid+1, end, idx, val); // right child
        
        tree[node] = tree[2*node] * tree[2*node+1];
        
    }

    int query(int node, int start, int end, int l, int r){
        // range completely outside
        if(r < start or end < l) return 1; // multiplication neutral

        // range completely inside
        if(l <= start and end <= r) return tree[node];
        
        // range partially inside and partially outside
        int mid = (start + end) >> 1;
        int p1 = query(2*node, start, mid, l, r);
        int p2 = query(2*node+1, mid+1, end, l, r);
        return (p1 * p2);
    }
\end{verbatim}


\subsubsection{SegTree Aditiva com Lazy Propagation}
\par Lazy propagation serve para atualizações em intervalos.
\par Código adaptado para range queries e range updates (lazy propagation)
\begin{verbatim}

    vector<ll> a(MAX), tree(4*MAX), lazy(4*MAX);

    // call build(1, 0, n-1);
    void build(int node, int start, int end){
        if (start == end) tree[node] = a[start];
        else{
            int mid = (start+end)>>1;
    
            build(2*node,start,mid);
            build(2*node+1,mid+1,end);
    
            tree[node] = tree[2*node] + tree[2*node+1];
        }
    }

    // call updateRange(1, 0, n-1, l, r, val);
    void updateRange(int node, int start, int end, int l, int r, int val){
        if (lazy[node]){
            tree[node] += (end-start+1) * lazy[node];
            if (start != end){
                lazy[2*node] += lazy[node];
                lazy[2*node+1] += lazy[node];
            }
            lazy[node] = 0;
        }
    
        if (start>end || start>r || end<l) return;
    
        if (start>=l && end<=r){
            tree[node] += (end-start+1) * val;
            if (start != end){
                lazy[2*node] += val;
                lazy[2*node+1] += val;
            }
            return;
        }
    
        int mid = (start+end) >> 1;
        updateRange(2*node, start, mid, l, r, val);
        updateRange(2*node+1, mid+1, end, l, r, val);
        tree[node] = tree[2*node] + tree[2*node+1];
    }

    // call queryRange(1, 0, n-1, l, r);
    ll queryRange(int node, int start, int end, int l, int r){
        if (start>end || start>r || end<l) return 0;
    
        if (lazy[node]){
            tree[node] += (end-start+1) * lazy[node];
            if (start != end){
                lazy[2*node] += lazy[node];
                lazy[2*node+1] += lazy[node];
            }
            lazy[node] = 0;
        }
    
        if (start>=l && end<=r) return tree[node];
    
        int mid = (start+end)>>1;
        ll p1 = queryRange(2*node, start, mid, l, r);
        ll p2 = queryRange(2*node+1, mid+1, end, l, r);
        return (p1+p2);
    }
\end{verbatim}
\subsection{Fenwick Tree (BIT)}

\par Uma estrutura que, de modo semelhante à SegTree, realiza operações de consulta e de atualização ambas em $O(\log n)$. OBS: a árvore bit é indexada a partir de 1.

\subsubsection{BIT aditiva}

\begin{verbatim}
    int n, bit[MAX];

    void setbit(int i, int delta){
        while (i <= n){
            bit[i] += delta;
            i += i&-i; // i&-i retorna a maior potencia de 2 que divide i
        }
    }

    int getbit(int i){
        int ans = 0;
        while(i){
            ans += bit[i];
            i -= i&-i; // subtrai maior potencia de 2 que divide i
        }
        return ans;
    }
\end{verbatim}

\subsection{Bitset}
    Maneira de armazenar uma string binária de forma eficiente, com complexidade O(n/32) e O(n/64). A utilização pode ser feita da mesma forma que vetores, com a condição da estrutura favorecer operações bit a bit.
\begin{verbatim}
bitset<n> nome(k); /*criar bitset de tamanho n (deve ser 
conhecido em tempo de compilação) que armazena o número k.*/
\end{verbatim}
    
\pagebreak

